\appendix
\renewcommand{\thesection}{\Alph{chapter}.\arabic{section}}
\renewcommand*{\thesatz}{\Alph{chapter}.\arabic{satz}}
\setcounter{section}{0}

% \begin{appendices}
% \setcounter{chapter}{2}

\chapter{Auxiliary results}
\manualmark \markboth{}{Auxiliary results}

%\begin{prop}[Complement of DPPs]
%Let \(\mathbf Y\) be a random subset that is distributed according to a DPP with marginal kernel \(K\in\mathbb R^{N\times N}\). Then we have
%\[\mathbb P(A\subseteq\mathbf Y^c) = \det\big( (I-K)_A\big). \]
%Hence the complement \(\mathbf Y^c\) is distributed according to a DPP with marginal kernel \(I-K\).
%\end{prop}
%\begin{proof}

%\end{proof}

%\begin{prop}[Normalisation]
%Let \(L\in\mathbb R^{N\times N}\) for \(N\in\mathbb N\). Then we have
%\[\sum_{A\subseteq [N]} \det(L_A) = \det(L + I) \]
%where \([N]\coloneqq\left\{ 1, \dots, N\right\}\).
%\end{prop}
%\begin{proof}

%\end{proof}

The only result we used during the thesis that we consider not general mathematical knowledge is the intersection theorem of Cantor. Since it is hard to find a reference of the general statement we quickly present the proof that can also be found on the \href{https://en.wikipedia.org/wiki/Cantor%27s_intersection_theorem}{Wikipedia entry}.%Wikipedia entry
% \url{https://en.wikipedia.org/wiki/Cantor%27s_intersection_theorem}.

\begin{theo}[Cantor�s intersection theorem]
Let \(\mathcal X\) be a topological Hausdorff space and let \(K_1\supseteq K_2\supseteq \dots\) be a sequence of descending, non empty compact sets. Then also the intersection
\[\bigcap_{n=1}^\infty K_n\]
is non empty.
\end{theo}
\begin{proof}
Assume that the intersection would be empty, and set \(U_n\coloneqq \mathcal X\setminus K_n\) which is open since \(K_n\) is closed as the compact subset of a Hausdorff space. Then \((U_n)_{n\in\mathbb N}\) is an open covering of \(K_1\) since we have
\[\bigcup_{n=1}^\infty U_n = \mathcal X\setminus\left( \bigcap_{n=1}^\infty K_n\right) = \mathcal X. \]
Hence we can select a finite subcover and obtain
\[K_1\subseteq \bigcup_{n=1}^N U_n = \mathcal X\setminus \left( \bigcap_{n=1}^N K_n\right) = \mathcal X\setminus K_N.\]
However since \(K_n\subseteq K_1\) this implies \(K_N = \varnothing\) which is a contradiction.
\end{proof}

\chapter{Generated code}
\manualmark \markboth{}{Generated code}

All the coding was done in R and the code of the sampling algorithm, the general MCMC methods and also the estimation of the log linearity constant will be provided here. % and also the learning algorithm of my toy example here.
%During the coding I tried to follow Google�s R Style Guide (https://google.github.io/styleguide/Rguide.xml).

% Some text

% \chapter{bla title}
% Some text
% \subsection{foo title}
% Some text

\section{Sampling algorithm}

\begin{lstlisting}[language=R, basicstyle=\footnotesize] % , commentstyle=\texttt] %, keywordstyle=\textrm]

# Implementation of the sampling algorithm as a function of the
# eigendecomposition of the elementary kernel L

SamplingDPP <- function (lambda, eigenvectors) {
  # First part of the algorithm, doing the selection of the eigenvectors
  N = length(lambda)
  J <- runif(N) <= lambda/(1 + lambda)
  k <- sum(J)
  V <- matrix(eigenvectors[, J], nrow=N)
  Y <- rep(0, k)
  
  # Second part of the algorithm, the big while loop
  while (k > 0) {
    # Calculating the weights and selecting an item i according to them
    wghts <- k^(-1) * rowSums(V^2)
    i <- sample(N, 1, prob=wghts)
    Y[k] <- i
    if (k == 1) break
    
    # Projecting e_i onto the span of V
    help <- V %*% V[i, ]
    help <- sum(help^2)^(-1/2) * help
    
    # Projecting the elements of V onto the subspace orthogonal to e_i
    V <- V - help %*% t(t(V) %*% help)
    
    # Orthonormalize V and set near zero entries to zero
    V[abs(V) < 10^(-9)] <- 0
    j <- 1
    while(j <= k) {
      help2 <- rep(0, N)
      m <- 1
        while (m <= j - 1) {
        help2 <- help2 + sum(V[, j] * V[, m]) * V[, m]
        m <- m + 1
      }
      V[, j] <- V[, j] - help2
      if (sum(V[, j]^2) > 0) {
        V[, j] <- sum(V[, j]^2)^(-1/2) * V[, j]
      }
      j <- j + 1
    }
    V[abs(V) < 10^(-9)] <- 0
    
    # Selecting a linear independent set in V
    k <- k - 1
    q <- qr(V)
    V <- matrix(V[, q$pivot[seq(k)]], ncol=k)
  }
  return(Y)
}
\end{lstlisting}

%\section{Points on the line}

%\begin{lstlisting}[language=R, basicstyle=\footnotesize]

%# NEEDS: sampling algorithm

%# In this example we sample points on a (discrete) line according to a DPP
%# We model L directly and via the quality-diversity decomposition. We plot and
%# compare the patterns to uncorrelated points i.e. to a Poisson point process.

%# Minimal example _____________________________________________________________
%n <- 3
%L <- matrix(c(2,1,0,1,2,0,0,0,2), nrow=n)

%# Points on a line ____________________________________________________________
%n <- 100
%L <- rep(0, n^2)
%for (i in 1:n) {
%  for (j in 1:n) {
%    L[(i - 1) * n + j] <- dnorm((i-j) * n^(-1/4))
%  }
%}
%L <- matrix(L, nrow=n)

%# Modelling phi and q _________________________________________________________
%# Points on the line.
%m <- 99  # 29
%n <- m + 1
%q <- rep(10, n)  # 0-1 sequences: rep(10^2, n)
%phi <- rep(0, n^2)
%for (i in 1:n) {
%  for (j in 1:n) {
%    phi[(i - 1) * n + j] <- dnorm((i - j) / 10)  # 0-1 sequences: devide by 2
%  }
%}
%phi <- matrix(phi, ncol=n)

%# Log linear quality for the points on the line _______________________________
%m <- 99
%n <- m + 1
%q <- rep(0, n)
%for (i in 1:n) {
%  q[i] <- 10^2 * sqrt(m) * exp(-0.2 * abs(i - 50.5))
%}
%phi <- rep(0, n^2)
%for (i in 1:n) {
%  for (j in 1:n) {
%    phi[(i - 1) * n + j] <- dnorm(2 * (i - j) / sqrt(m))
%  }
%}
%phi <- matrix(phi, ncol=n)

%# General part, define L ______________________________________________________
%for (i in 1:n) {
%  phi[, i] <- sum(phi[, i]^2)^(-1/2) * phi[, i]
%}
%S <- t(phi) %*% phi
%time <- proc.time()
%L <- t(q * S) * q
%proc.time() - time

%# Compute the eigendecomposition, set near zero eigenvalues to zero and
%# set up poisson point process with same expected cardinality _________________
%time <- proc.time()
%edc <- eigen(L)
%lambda <- edc$values
%lambda[lambda < 10^(-9)] <- 0
%mean <- sum(lambda / (1 + lambda))
%eigenvectors <- edc$vectors
%lambda2 <- rep(mean / n / (1 - mean / n), n)
%eigenvectors2 <- diag(rep(1, n))
%proc.time() - time

%# Sample and plot things ______________________________________________________
%# Minimal example

%# 0-1 sequences
%x <- sort(SamplingDPP(lambda, eigenvectors))
%as.integer(1:n %in% x)
%y <- sort(SamplingDPP(lambda2, eigenvectors2))
%as.integer(1:n %in% y)

%# Sample from both point processes and plot the points on the line
%pointsDPP <- SamplingDPP(lambda, eigenvectors)
%pointsPoisson <- SamplingDPP(lambda2, eigenvectors2)
%plot(rep(1, length(pointsDPP)), pointsDPP,
%     ylim=c(1, n), xlim=c(.4, 3.2), xaxt='n', ylab="Points", xlab="")
%points(rep(2, length(pointsPoisson)), pointsPoisson, pch=5)
%legend("topright", inset=.05, legend=c("DPP", "Poisson"), pch=c(1, 5))

%\end{lstlisting}

%\section{Points in the square}

%\begin{lstlisting}[language=R, basicstyle=\footnotesize]

%# NEEDS: sampling algorithm

%# In this example we sample points on a two dimensional grid according to a DPP
%# We model L directly and via the quality-diversity decomposition including
%# different dimensions D for the feature vectors phi. We plot and compare the
%# patterns to uncorrelated points i.e. to a Poisson point process.

%# Define the coordinates of a point ___________________________________________
%CoordinatesNew <- function(i, n) {
%  y1 <- floor((i - 1) / (n + 1))
%  x1 <- i - 1 - (n + 1) * y1
%  return (t(matrix(c(x1, y1)/n, nrow=length(i))))
%}
%DistanceNew <- function (i, j, n, d) {
%  return (sqrt(colSums((CoordinatesNew(i, n) - CoordinatesNew(j, d))^2)))
%}

%# Direct modelling of L _______________________________________________________
%m <- 19
%n <- (m + 1)^2
%L <- rep(0, n^2)
%for (i in 1:n) {
%  for (j in 1:n) {
%    L[(i - 1) * n + j] = n^2 * dnorm(Distance(i, j, m))
%  }
%}
%L <- matrix(L, nrow=n)

%# Modelling phi and q _________________________________________________________
%# Points in the square.
%m <- 19
%n <- (m + 1)^2
%q <- rep(sqrt(m), n)
%x <- ceiling(1:n^2 / n)
%y <- rep(1:n, n)
%time <- proc.time()
%phi <- dnorm(sqrt(m) *matrix(DistanceNew(x, y, m, m), n))
%proc.time() - time

%# Quality diversity decomposition with small D ________________________________
%d <- 25
%q <- rep(10^5 * sqrt(m), n)
%x <- ceiling(1:(n*d) / d)
%y <- rep(1:d, n)
%time <- proc.time()
%phi <- dnorm(2 * sqrt(m) * matrix(DistanceNew(x, y, m, sqrt(d) - 1), ncol=n))
%proc.time() - time

%# Log linear quality for the points in the square _____________________________
%m <- 39
%n <- (m + 1)^2
%q <- exp(-6 * DistanceNew(rep(5, n), 1:n, 2, m) + log(sqrt(m)))
%x <- ceiling(1:n^2 / n)
%y <- rep(1:n, n)
%time <- proc.time()
%phi <- dnorm(2 * sqrt(m) * matrix(DistanceNew(x, y, m, m), n))
%proc.time() - time


%# General part, defining L ____________________________________________________
%# d <- length(phi) / n
%for (i in 1:n) {
%  phi[, i] <- sum(phi[, i]^2)^(-1/2) * phi[, i]
%}
%S <- t(phi) %*% phi
%# B <- t(phi) * q
%time <- proc.time()
%L <- t(t(q * S) * q)  # B %*% t(B)
%proc.time() - time

%# Compute the eigendecomposition, set near zero eigenvalues to zero and
%# set up poisson point process with same expected cardinality _________________
%time <- proc.time()
%edc <- eigen(L)
%lambda <- edc$values
%lambda[abs(lambda) < 10^(-9)] <- 0
%mean <- sum(lambda / (1 + lambda))
%eigenvectors <- edc$vectors
%lambda2 <- rep(mean / n / (1 - mean / n), n)
%eigenvectors2 <- diag(rep(1, n))
%proc.time() - time

%# Sample from both point processes and plot the points in the square __________
%# par(mfrow = c(1,1))
%time <- proc.time()
%dataDPP <- sort(SamplingDPP(lambda, eigenvectors))
%pointsDPP <- t(CoordinatesNew(dataDPP, m))
%plot(pointsDPP, xlim=0:1, ylim=0:1, xlab="", ylab="", xaxt='n', yaxt='n', asp=1)
%proc.time() - time
%dataPoisson <- sort(SamplingDPP(lambda2, eigenvectors2))
%pointsPoisson <- t(CoordinatesNew(dataPoisson, m))
%plot(pointsPoisson, xlim=0:1, ylim=0:1, xlab="", ylab="",
%					xaxt='n', yaxt='n', asp=1)

%# Remove all objects apart from functions
%rm(list = setdiff(ls(), lsf.str()))
%\end{lstlisting}

%\section{Toy learning example}

%\begin{lstlisting}[language=R, basicstyle=\footnotesize]
%# NEEDS: Sampling algorithm, declaration of the points in the square
%# TODO: Maybe do the gradient descent directly over the representation
%# od the gradient

%# With this toy example we aim to perform the first learning of paramters
%# associated to a kernel of a DPP. More precisely we will generate our own
%# data of points on a two dimensional grid with a log linear quality model
%# and aim to estimate the log linearity parameter.

%# Generation of data
%time <- proc.time()
%T <- 30
%data <- rep(list(0), T)
%for (i in 1:T) {
%  data[[i]] <- sort(SamplingDPP(lambda, eigenvectors))
%}
%proc.time() - time

%# Define the quality q, L, the feature sum and the loss in dependency of the
%# parameter theta
%Quality <- function(theta) {
%  return(exp(theta[1] * DistanceNew(rep(5, n), 1:n, 2, m) + theta[2]))
%}
%LFunction <- function(theta) {
%  return(t(t(Quality(theta) * S) * Quality(theta)))
%} 
%Feature <- function(A) {
%  # return(sum(DistanceNew(rep(5, length(A)), A, 2, m)))
%  return(c(sum(DistanceNew(rep(5, length(A)), A, 2, m)), length(A)))
%}
%Loss <- function(theta) {
%  T <- length(data)
%  # Sum this over all data entries
%  x <- 0
%  for (i in 1:T) {
%    A <- data[[i]]
%    x <- x + 2 * sum(theta * Feature(A)) + log(det(matrix(S[A, A], length(A))))
%  }
%  return(- x + T * log(det(diag(rep(1, n)) + LFunction(theta))))
%}

%# Parameter estimations
%time <- proc.time()
%sol <- nlm(Loss, c(-3, 0))
%proc.time() - time
%sol$estimate

%# Remove all objects apart from functions
%rm(list = setdiff(ls(), lsf.str()))
%\end{lstlisting}

\section{Implementation of the MCMC methods}

\begin{lstlisting}[language=R, basicstyle=\footnotesize]
# First we implement the Metropolis-Hastings algorithm. We implement the
# propose and reject step. We use a Gaussian as a proposal with covariance
# matrix alpha times the identity.
# Load library for multivariate normal.
library(MASS)
Metropolis <- function(x, f, alpha=1){
  d <- length(x)
  if (length(alpha) == 1) {
    alpha <- diag(rep(alpha, d), d)
  }
  y <- mvrnorm(1, x, alpha)
  z <- f(y)
  if (is.nan(z) || runif(1) * f(x) > z) y <- x
  return(y)
}

# Now we turn towards slice sampling. Proposing a random interval that includes
# the slice. We use an exponential random variable to define the width of the
# interval.
RandomInterval <- function (x, y, f, alpha=1) {
  c <- f(x)
  # We make the interval the same length in every dimension.
  a <- rexp(1, rate=alpha)  # rexp(length(x), rate=alpha)
  b <- rexp(1, rate=alpha)  # rexp(length(x), rate=alpha)
  # One can check both endpoints simultaneously to avoid the need of two loops.
  while (TRUE) {
    help <- f(x - a)
    if (is.nan(help) || help < c * y) break
    a <- 2 * a
  }
  while (TRUE) {
    help <- f(x + b)
    if (is.nan(help) || help < c * y) break
    b <- 2 * b
  }
  return(matrix(c(x - a, x + b), length(x)))
}
# Doing a single slice sample.
SliceSampling <- function (x, f, alpha=1) {
  d <- length(x)
  a <- f(x)
  y <- runif(1)
  c <- RandomInterval(x, y, f, alpha)
  z <- runif(d, c[, 1], c[, 2])  # runif(1, -4, 4)
  while (TRUE) {
    help <- f(z)
    if (is.nan(help) || help < a * y) z <- runif(d, c[, 1], c[, 2])
    else break
  }
  return(z)
}

# Implementing the MCMC method. The function needs the unnormalised density f,
# a starting value x0, sample size T whether it should be MH or Slice Sampling
# and the parameter alpha, which either specifies the variance of the proposal
# which is multivariate normal or the rate of the exponential random variable
# which defines the thickness of the random interval.
MCMC <- function (f, x0, T=10^3, MH=TRUE, alpha=1) {
  d <- length(x0)
  x <- matrix(rep(x0, T), d)
  if (MH) {
    for (t in 2:T) x[, t] <- Metropolis(x[, t-1], f, alpha) 
  }
  else {
    # Check whether starting value is impossible. In this case the slice is the
    # whole space and hence the endpoints of the random interval will diverge.
    while (is.nan(f(x0)) || f(x0)==0) {
      x0 <- mvrnorm(1, x0, diag(rep(alpha, d), d))
    }
    x[, 1] <- x0
    for (t in 2:T) x[, t] <- SliceSampling(x[, t-1], f, alpha)
  }
  return(x)
}
\end{lstlisting}
%# One dimensional case, making nice pictures also.
%# Define the unnormalised density we want to sample from.
%target <- function(x){
%  sin(x)^2 * sin(2 * x)^2 * exp(-x^2/2)
%}
%# Load library for multidimensional integration to compute the normalisation
%# constant.
%library(cubature)
%Z <- hcubature(target, -20, 20)[[1]]

%time <- proc.time()
%x <- MCMC(target, 1, MH=FALSE, 2 * 10^4, alpha=10)
%proc.time() - time
%hist(x, breaks=seq(min(x), max(x), length=100), freq=FALSE, ylim=c(0, 0.8),
%     main="Histogram for alpha=10", xlab="x")
%y <- seq(min(x), max(x), length=500)
%z <- target(y) / Z
%lines(y, z, col="red", lwd=2)
%acf(x[1:1000], 100, main="Autocorrelation for alpha=10")
%plot(x[1,1:2000], type="l", xlab="Step of the chain",
%     ylab="State of the chain", main="MH random walk for alpha=100")
%# Calculating the acceptence rate for the MH algorithm; around 25% is desired
%sum(x[-1] != x[1:(length(x) - 1)])/(length(x) - 1)

%# Two dimensional toy example with a similar density. The points are plotted
%# and a heat map is created which also shows the marginal densities.
%target2 <- function(x){
%  sin(x[1])^2 * sin(2 * x[2])^2 * dnorm(x[1]) * dnorm(x[2])
%}
%x <- MCMC(target2, c(1, 1), MH=TRUE, 10^4, alpha=3)
%plot(t(x), pch=16, col='black', cex=0.5)
%# Color housekeeping and loading library for heat map.
%library(RColorBrewer)
%rf <- colorRampPalette(rev(brewer.pal(11,'Spectral')))
%r <- rf(32)
%# library(MASS)
%k <- kde2d(x[1, ], x[2, ], n=600, lims = c(-4.3, 4, -4, 4))
%image(k, col=r, xlim=c(-4.3, 4), ylim=c(-4, 4))
%# Adding the marginal densities.
%y1 <- seq(-4, 4, length=500)
%z1 <- sin(2 * y1)^2 * dnorm(y1) * 3
%lines(z1 - rep(3.8, 500), y1, col="red", lwd=2)
%y2 <- seq(-4.3, 4, length=500)
%z2 <- sin(y2)^2 * dnorm(y2) * 3
%lines(y2, z2 - rep(3.5, 500), col="red", lwd=2)

\section{MLE and Bayesian estimation of the log linearity constant}

\begin{lstlisting}[language=R, basicstyle=\footnotesize]
# NEEDS: SamplingDPP, defineS, example of DPP on a two dimensional grid with
# log linear qualities including the eigendecompositon of L.

# With this toy example we aim to perform the first learning of paramters
# associated to a kernel of a DPP. More precisely we will generate our own
# data of points on a two dimensional grid with a log linear quality model
# and aim to estimate the log linearity parameter.

# Generation of the data
T <- 20
data <- rep(list(0), T)
for (i in 1:T) {
  data[[i]] <- sort(SamplingDPP(lambda, eigenvectors))
}
# Exporting the data into a .txt file.
write.table(toString(data), "mydata.txt", sep="\n")

# Define the quality q, L, the feature sum and the loss in dependency of the
# parameter theta.
Quality <- function(theta) {
  return(exp(theta[1] * DistanceNew(rep(5, n), 1:n, 2, m) + theta[2]))
}
LFunction <- function(theta) {
  return(t(t(Quality(theta) * S) * Quality(theta)))
} 
# Define the sum of the diversity features over a set A.
Feature <- function(A) {
  return(c(sum(DistanceNew(rep(5, length(A)), A, 2, m)), length(A)))
}
# Define the observation probability and the log likelihood function.
ObservationProbability <- function (theta, data) {
  T <- length(data)
  x <- 1
  a <- det(diag(rep(1, n)) + LFunction(theta))
  for (t in 1:T) {
    A <- data[[t]]
    x <- x * exp(2 * sum(theta * Feature(A))) * det(S[A, A]) / a * 10^45
  }
  return(x)
}
LogLikelihood <- function(theta, data) {
  return(-log(ObservationProbability(theta, data)))
}
RegLogLikelihood <- function(theta, data) {
  return(LogLikelihood(theta, data) + sum(theta^2) / 16)
}
NewRegLogLikelihood <- function(theta, data) {
  return(LogLikelihood(theta, data) + sum(theta^2) / 2)
}

# Maximum likelihood estimation of the log linearity constant theta.
sol <- nlm(LogLikelihood, c(-8, 6))
mle <- sol$estimate
mle

# Learning curve of the MLE
T <- 30
learningCurve <- matrix(rep(0, 2 * T), 2)
regLearningCurve <- matrix(rep(0, 2 * T), 2)
NewRegLearningCurve <- matrix(rep(0, 2 * T), 2)
target <- function(theta) {
  return(LogLikelihood(theta, data))
}
regTarget <- function(theta) {
  return(RegLogLikelihood(theta, data))
}
NewRegTarget <- function(theta) {
  return(NewRegLogLikelihood(theta, data))
}
for (i in 1:T) {
  if (i > 1) {
    help <- data
  }
  data <- rep(list(0), i)
  if (i > 1) {
    for (j in 1:(i - 1)) {
      data[[j]] <- help[[j]]
    } 
  }
  data[[i]] <- sort(SamplingDPP(lambda, eigenvectors))
  learningCurve[, i] <- nlm(target, c(-10, 6))$estimate
  print(learningCurve[, i])
  regLearningCurve[, i] <- nlm(regTarget, c(-10, 6))$estimate
  print(regLearningCurve[, i])
  NewRegLearningCurve[, i] <- nlm(NewRegTarget, c(-10, 6))$estimate
  print(NewRegLearningCurve[, i])
}
plot(NewRegLearningCurve[1, 1:30], xlab="Number of samples", ylab="Estimate
     for first parameter", 
     pch=5, ylim=c(-10.85, -5.5))
points(learningCurve[1, 1:30], pch=16)
points(regLearningCurve[1, 1:30], pch=1)
lines(c(0, 45), c(-10, -10), col="red")
legend("topright", inset=.05, legend=c("MLE", "Weak regularisation", "Strong
                                       regularisation")
       , pch=c(16, 1, 5))
plot(NewRegLearningCurve[2, 1:30], xlab="Number of samples", ylab="Estimate for
     second parameter", 
     pch=5, ylim=c(2.8, 7))
points(learningCurve[2, 1:30], pch=16)
points(regLearningCurve[2, 1:30], pch=1)
lines(c(0, 45), c(6, 6), col="red")
legend("bottomright", inset=.05, legend=c("MLE", "Weak regularisation", "Strong
                                          regularisation"),
       pch=c(16, 1, 5))

# NEEDS: MCMC algorithm.

# We want to introduce the first example of Bayesian parameter estimation for
# DPPs. We start by estimating the log linearity constant of the qualities - a
# parameter for which we've already successfully done MLE.

# Run MCMC and create a plot that also shows the coordinates of the MLE.
# Putting a centered Gaussian as a prior.
target <- function(theta) {
  x <- exp(-sum(theta^2)  / 2^4) * ObservationProbability(theta)
  return(x)
}

# Sampling a 200 samples to find a reasonable starting point. Agressitivity of
# the MH is adjusted so that a reasonable acceptence rate is obtained.
x <- MCMC(target, c(0, 0), MH=TRUE, 10^2, alpha=10)
plot(t(x), pch=16, col='black', cex=0.5, xlab="theta1", ylab="theta2")
points(mle[1], mle[2], pch=4, lwd=3, col="green")
points(mean(x[1, 100:200]), mean(x[2, 100:200]), pch=4, lwd=3, col="red")
# Calculating the acceptence rate for the MH algorithm; 25-75% is desired.
sum(x[, -1] != x[, 1:(length(x)/2 - 1)])/(length(x) - 2)
# Plot the autocorrelation function.
acf(x[1, 1:100], 40)

# Second burn in period consisting of 10^3 samples which will be used to tune
# the proposal. The parameter alpha is adjusted such that a reasonable
# acceptance rate is obtained.
x2 <- MCMC(target, c(mean(x[1, 100:200]), mean(x[2, 100:200])), MH=TRUE, 10^3,
           alpha=2)
plot(t(x2[,250:10^3]), pch=16, col='black', cex=0.5, xlab="theta1",
     ylab="theta2")
points(mle[1], mle[2], pch=4, lwd=3, col="green")
# Calculating the acceptence rate for the MH algorithm; 25-75% is desired.
sum(x2[, -1] != x2[, 1:(length(x2)/2 - 1)])/(length(x2) - 2)
# Plot the autocorrelation function.
acf(x2[1,1:1000], 100, main="Autocorrelation for alpha=10")

# Doing PCA with the first 100 samples in order to tune the proposal.
library(stats)
pc <- prcomp(t(x2[,250:10^3]))
sd <- t(pc[[2]]) %*% diag(c(pc[[1]][[1]], pc[[1]][[2]])^2) %*% pc[[2]]
# Let the main MCMC run with 10^4 samples. Also a nice plot is created.
xnew <- MCMC(target, c(mean(x[1, ]), mean(x[2, ])), MH=TRUE, 10^4, alpha=sd)
plot(t(xnew))
library(RColorBrewer)
rf <- colorRampPalette(rev(brewer.pal(11,'Spectral')))
r <- rf(64)
k <- kde2d(xnew[1, ], xnew[2, ], n=1000, lims = c(-12, -3, -1, 8))
image(k, col=r, xlim=c(-12, -8.5), ylim=c(5, 7))
points(mle[1], mle[2], pch=4, lwd=3, col="green")
points(-10, 6, pch=4, lwd=3, col="white")
# Calculating the acceptence rate for the MH algorithm; 25-75% is desired.
sum(xnew[, -1] != xnew[, 1:(length(xnew)/2 - 1)])/(length(xnew) - 2)
# Plot the autocorrelation function.
acf(xnew[1,1:1000], 100, main="Autocorrelation for alpha=10")

# Gelman-Rubin diagnostic:
# Run a second chain:
newStart <- mvrnorm(1, start, 2 * sd)
xnew2 <- MCMC(target, newStart, MH=TRUE, 10^4, alpha=sd)
install.packages("coda")
library(coda)
combinedchains <- mcmc.list(mcmc(data=t(xnew)), mcmc(data=t(xnew2)))
gelman.diag(combinedchains)
gelman.plot(combinedchains, ylim=c(1, 1.5), ylab="R-hat value", lwd=1.5)

# Stability analysis
# Define Poisson point process
poisson <- function(p, n) {
  return((1:n)[runif(n) < p])
}
poisson(0.2, 20)
symmdiff <- function(a, b) {
  return(union(setdiff(a, b), setdiff(b, a)))
}

T <- 8
perturbedData1 <- rep(list(0), T)
sampleSize <- 8
for (i in 1:sampleSize) {
  # Generation of data
  for (i in 1:T) {
    perturbedData1[[i]] <- sort(SamplingDPP(lambda, eigenvectors))
  }
  for (i in 1:T) {
    perturbedData1[[i]] <- symmdiff(perturbedData1[[i]], poisson(1/400, n))
  }
  # Estimators
  target <- function(theta) {
    return(LogLikelihood(theta, perturbedData1))
  }
  print(nlm(target, c(-10, 6.5))$estimate)
  target <- function(theta) {
    return(RegLogLikelihood(theta, perturbedData1))
  }
  print(nlm(target, c(-10, 6.5))$estimate)
  target <- function(theta) {
    return(NewRegLogLikelihood(theta, perturbedData1))
  }
  print(nlm(target, c(-10, 6.5))$estimate)
}
\end{lstlisting}