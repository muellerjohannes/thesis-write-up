\chapter{Bayesian parameter estimation and Markov chain Monte Carlo methods}


So far we have seen two different estimation techniques for the parameters of DPPs. Although we proved that they provide reasonable estimators in the sense that they are consistent, they have some drawbacks. For example the MLEs for the different parameters do not exist in general, let alone that they are impossible to compute in practice. Further all of the estimators presented so far are point estimators, i.e. they return a single value for the desired parameter. Obviously this does not allow to capture any uncertainties that the estimation of the parameter has. Those are some reasons to consider the Bayesian approach of parameter estimation where the goal is to give a distribution -- called the posterior -- of the parameter that should be estimated instead of a single value. This can also help to overcome some -- maybe even all of the problems mentioned above.

At first we will present the general concept of Bayesian parameter estimation and will then turn towards the question of computability of the posterior distribution. For this we will follow the approach of \cite{affandi2014learning} and turn towards the popular Markov chain Monte Carlo (MCMC) methods. We quickly explain their philosophy and how they can be used to approximate the posterior distribution of the parameter one wishes to estimate.

\section{Bayesian approach to parameter estimation}

For the introduction of the general Bayesian setup we pursue like in \cite{rice2006mathematical}. Just like in the case of maximum likelihood estimation we want to estimate a parameter \(\theta\in\Theta\) based on
 some realisations \(x = (x_1, \dots, x_n)\) of random variables \(X = (X_1, \dots, X_n)\). This time, however, we are not interested in returning a single value \(\theta\) because this would be a vast simplification of the stochastic nature of the estimator. We rather want to obtain a probability distribution over whole parameter space \(\Theta\) that indicates how likely the parameters are to have caused the observed data. In order to present the procedure we will introduce the frame we will work in.

\begin{emp}[Setting]
Let \(\Theta\) be a measurable space and \(\nu\) be a measure on \(\Theta\). Let \(f_\Theta\colon \Theta\to[0, \infty]\) be a probability density with respect to \(\nu\), i.e.
\[\int\limits_{\Theta} f_\Theta(\theta)\nu(\mathrm{d}\theta) = 1\]
which we will call the \emph{prior} distribution of the parameter \(\theta\).\footnote{The requirement of \(f\) being a probability density can easily be loosened. In fact if it has finite integral it is obvious that the normalisation cancels in the definition \eqref{post} of the posterior and even if it has infinite integral, \eqref{post} might still give a probability density.} Further let
\[\mathcal F = \Big\{ f_{X| \Theta}(\cdot |\theta) \;\big\lvert\; \theta\in\Theta\Big\}\]
by a family of probability densities with respect to \(\mu^n\coloneqq\prod_{i=1}^n\mu(\mathrm d x_i)\).
\end{emp}

Usually the prior distribution will encode some perceptions or prior knowledge we might have of the parameter. For example if we are trying to estimate a physical constant that we know has to be positive, then it is reasonable to select a prior that has its whole mass on the positive real line. However, there is no clear set of rules how one can select a suitable prior to a given problem. 

The density \(f_{X|\Theta}(x|\theta)\) describes how likely the observations are under the parameter \(\theta\) and we want to find an expression of how likely the parameter \(\theta\) is under the observations \(x\). In order to obtain this, we will work with the joint density
\[f_{X, \Theta}(x, \theta) = f_{X|\Theta}(x|\theta) f_\Theta(\theta) \quad \text{with respect to } \mu^n \times \nu \]
and condition this onto \(x\). This yields
\begin{equation}\label{post}
\begin{split}
f_{\Theta| X}(\theta|x) = \frac{f_{X, \Theta}(x, \theta)}{\int_{\Theta} f_{X, \Theta}(x, \theta)\nu(\mathrm{d}\theta)} = \frac{f_{X|\Theta}(x| \theta) f_\Theta(\theta)}{\int_{\Theta} f_{X, \Theta}(x, \theta)\nu(\mathrm{d}\theta)}.
\end{split}
\end{equation}

\begin{defi}[Posterior distribution]
The density \(f_{\Theta|X}\) is called the \emph{posterior distribution} of the parameter \(\theta\) given the data \(x\). Further we call the normalisation constant
\[f(x|\mathcal F) \coloneqq \int_{\Theta} f_{X, \Theta}(x, \theta)\nu(\mathrm{d}\theta) \]
the total probability of the data \(x\) under the model \(\mathcal F\).
\end{defi}

First we will convince ourselves that the approach of calculating a posterior distribution is a generalisation of the MLE in a lot of cases.

\begin{emp}[Comparison to MLE]
Maybe one feels slightly uncomfortable with the need to choose a prior distribution and it turns out that this is in fact a difficult step that has to be taken with a certain amount of care. However, we could pretend for one moment to be completely ignorant in the sense that we do not know anything about the parameter and hence we don�t feel in the position to propose a reasonable prior. Then we could simply choose the uniform distribution as a prior -- given it exists\footnote{Even it doesn�t one can still define the prior density to be constant and hope that the posterior is a probability density.} -- and would obtain
\[f_{\Theta| X}(\theta|x) \propto f_{X|\Theta}(x|\theta). \]
Hence we can regain the MLE from our posterior distribution since it is just the mode, i.e. the maximiser of the posterior density. This relation to the MLE can be seen in Figure \ref{fig:4.1}. Hence, the Bayesian approach is a more general tool than MLE and allows to capture the randomness of the parameter \(\theta\). This is desirable since  we have seen that the mode is not always a very typical outcome of a random variable.
\begin{figure}[h!]
	\centering
	\includegraphics[width=0.6\textwidth]{figures/heatmap-log-linearity-SliceSampling-new-3}
	\caption{Approximated posterior density of the two dimensional log linearity constant of a two dimensional DPP with a uniform distribution as a prior. The MLE estimator is marked green and is at the mode of the distribution.}
	\label{fig:4.1}
\end{figure}

A further advantage over the MLE is that it might be possible to computationally approximate the posterior density, but not the MLE. This is typically the case if the log likelihood function is not concave, like in the setting of the MLE of the whole elementary kernel \(L\). In fact the only hard step in the calculation of the posterior \eqref{post} is the computation of the normalisation constant
\[\int\limits_{\Theta} f_{X,\Theta}(x, \theta) \nu(\mathrm{d}\theta). \]
This can often not be performed efficiently but the Markov chain Monte Carlo methods introduced later will yield an approximation of the posterior without the need to compute the normalisation constant.
\end{emp}

\begin{emp}[Regularisation through the prior]
The prior density is very closely related to the regulariser introduced in the section about maximum likelihood estimation. In fact the mode of the posterior \(f_{\Theta|X}\) is nothing else but the maximiser of 
\[\log(f_{\Theta|X}) = \mathcal L + R\]
where \(R = \log(f_\Theta)\) and hence nothing else but a regularised MLE. In fact if \(R\) is a regularisation and \(\exp(R)\) is integrable with respect to \(\mu\), then one can choose \(f_\Theta\propto \exp(R)\) as a prior density. Hence the proposition of a prior and a regulariser are equivalent in a wide variety of cases, but again, the posterior density encodes much more information than just the location of its mode which is the regularised MLE.
\end{emp}

\begin{emp}[Bayesian approach without prior]
We have seen that the prior is nothing else but a regularisation of the likelihood and since MLE can be carried out without regularisation it is natural to ask whether the Bayesian approach works without a prior. We have seen that if there is a uniform distribution on the parameter space, the unregularised MLE corresponds to the uniform distribution as a prior. So the question is what changes if we propose \(f_\Theta = 1\) as a prior if there is no uniform distribution, or more generally what happens if the prior \(f_\Theta\) has infinite integral.

In this case, the unnormalised posterior distribution 
\begin{equation}\label{genpost}
\mathfrak{f}(\theta) = f_{X|\Theta}(x| \theta) f_\Theta(\theta)
\end{equation}
might not have finite integral and can therefore not be normalised. Hence the posterior can not be seen as a probability density, but the generalised form \eqref{genpost} still exists. If we use the constant prior \(f_\Theta=1\), the generalised posterior \(\mathfrak{f}(\theta)\) is just the observation probability (or density if \(\mathcal X\) is continuous) of the data \(x\) under the parameter \(\theta\).
\end{emp}

\subsubsection*{Expression of the posterior for DPPs}

Now we will express the posterior in the case of DPPs under the following conditions.

\begin{emp}[Setting]
Let \((\Theta, \nu)\) be a measure space and \(L(\theta)\in\mathbb R^{N\times N}_{\text{sym}, +}\) be an elementary kernel for every \(\theta\in\Theta\). Further we assume that we have independent realisations \(Y_1, \dots, Y_n\) of a  \(L\)-ensemble.
\end{emp}

Typically the parametrisations \(\theta\mapsto L(\theta)\) will be one of the three parametric models in III.2.1, i.e. \(\theta\) will either be the whole kernel itself, the quality vector or the log linearity constant of the qualities and \(L(\theta)\) the associated elementary kernel.

The independence relation leads to a factorisation of the density and we obtain the following expression for the posterior density
\begin{equation}\label{post2}
f_{\Theta|\mathbf Y^n}(\theta|Y_1, \dots, Y_n) \propto f_\Theta(\theta) \prod_{i=1}^n f_{\mathbf Y|\Theta}(Y_i|\theta) = f_\Theta(\theta) \prod_{i=1}^n \frac{\det(L(\theta)_{Y_i})}{\det(L(\theta) + I)}.
\end{equation}

Unfortunately the normalisation constant
\begin{equation}\label{norma}
\int\limits_{\Theta} f_{\Theta|\mathbf Y^n}(\theta|Y_1, \dots, Y_n)\nu(\mathrm{d}\theta) = \int\limits_{\Theta} f_\Theta(\theta) \prod_{i=1}^n \frac{\det(L(\theta)_{Y_i})}{\det(L(\theta) + I)}\nu(\mathrm{d}\theta)
\end{equation}
can neither be computed analytically nor numerically in an efficient way since the evaluation of this density involves the computation of the determinant of a \(N\times N\) matrix. This problem can be solved through the powerful methods of Markov chain Monte Carlo simulation that allow to approximate a distribution with only the knowledge of its unnormalised density. 

\subsubsection{Model selection using the Bayes factor}

In this paragraph we will quickly touch on how the Bayesian approach can be used to compare two different models, i.e. two different parametric families \(\mathcal F_1\) and \(\mathcal F_2\) including two different priors \(f_{\Theta_1}\) and \(f_{\Theta_2}\). For this we will work in the following setup.

\begin{emp}[Setting]
Let \(\Theta_1, \Theta_2\) be measurable spaces and \(\nu_i\)  measures on \(\Theta_i\) for \(i=1, 2\). Let \(f_{\Theta_i}\colon \Theta_i\to[0, \infty]\) be probability densities with respect to \(\nu_i\), i.e.
\[\int\limits_{\Theta_i} f_{\Theta_i}(\theta)\nu_i(\mathrm{d}\theta) = 1 \quad \text{for } i = 1, 2.\]
Further let
\[\mathcal F_i = \Big\{ f_{X| \Theta_i}(\cdot |\theta) \;\big\lvert\; \theta\in\Theta_i\Big\}\]
by a family of probability densities with respect to \(\mu^n\coloneqq\prod_{i=1}^n\mu(\mathrm d x_i)\).
\end{emp}

The goal is now two compare which model \(\mathcal F_i\) in combination with the corresponding prior describes the phenomenon better given some data \(x\). For this we follow \cite{kass1995bayes} and introduce the \emph{Bayes factor} of the two models given the data \(x\) through
\[K \coloneqq K(\mathcal F_1, \mathcal F_2| x) \coloneqq\frac{f(x|\mathcal F_1)}{f(x|\mathcal F_2)} = \frac{\int_{\Theta_1} f_{X| \Theta_1}(x| \theta) f_{\Theta_1}(\theta)\nu_1(\mathrm{d}\theta)}{\int_{\Theta_2} f_{X| \Theta_2}(x| \theta) f_{\Theta_2}(\theta)\nu_2(\mathrm{d}\theta)}. \]
This is nothing but the ratio of the total probabilities of the data \(x\) under the respective models. If this ratio is big, the model \(\mathcal F_1\) including its prior can be seen as a better description of the data compared to the second model. There is no clear definition on when the ratio can be seen as big enough to say this, but the following guidelines in Table \ref{tab:BayesFactor} were proposed in \cite{kass1995bayes}.

\begin{table}
\centering
\begin{tabular}{ |c|c| } 
 \hline
  Value for \(K\) & Interpretation \\ \hline
 \(1 - 3.2\) & Only worth a bare mention \\ \hline
 \(3.2 - 10\) & Substantial \\ \hline
 \(10 - 100\) & Strong \\ \hline
 \(>100\) & Decisive \\ \hline
\end{tabular}
\caption{Interpretation of how strongly different Bayes factors imply that one model a better description of the data than the other one.} \label{tab:BayesFactor}
\end{table}

\section{Markov chain Monte Carlo methods}

The method of Markov chain Monte Carlo (MCMC) simulation arose almost as early as the Monte Carlo\footnote{A legend has it that the name Monte Carlo was given to the work of von Neumann and Ulam by a colleague referring to Ulam�s uncle who lost a significant amount of money gambling in the Monte Carlo casino in Monaco.} simulations itself and since then a rich theory has been established and a broad range of applications have been found. However, we can only give a short overview over the basic principles and refer to \cite{meyn2012markov} for an introduction of Markov chain theory and to \cite{robert2013monte} for a survey on (Markov chain) Monte Carlo methods. 

Our motivation for the study of MCMC methods was to obtain an approximation of a distribution \(\pi\) under the knowledge of its unnormalised density. In the nutshell the idea is to construct an ergodic Markov chain \((X_n)_{n\in\mathbb N}\) with stationary distribution \(\pi\), i.e. such that one has
\[ \hat{\mathbb P}_n =\frac1n \sum_{i=1}^n \delta_{X_n} \xlongrightarrow{n\to\infty} \pi \]
almost surely in the weak sense. 
This Markov chain can then be simulated using Monte Carlo methods and the associated empirical measures \(\hat{\mathbb P}_n\) will be approximations of \(\pi\). However, to explain this in more detail we need to recapture some notions of Markov chains.

\subsection{Reminder on Markov chains}

We will provide an extremely short presentation of only those results that we will use to explain the core of MCMC methods. However, this will not contain any proofs and hence it can not replace the study of the already mentioned text books.

Let in the following \((\mathcal X, \mathcal B(\mathcal X))\) be a measurable space. This will typically be a topological space with its Borel algebra later.


\begin{defi}[Markov chain]
\begin{enumerate}
\item A \emph{transition kernel} is a function \[K\colon\mathcal X\times\mathcal B(\mathcal X)\to[0, 1]\] such that
\begin{enumerate}
\item \(K(x, \cdot)\) is a probability measure for every \(x\in\mathcal X\) and
\item \(K(\cdot, A)\) is measurable for every \(A\in\mathcal B(\mathcal X)\).
\end{enumerate}
\item A \emph{Markov chain} with values in \(\mathcal X\) and transition kernel \(K\) is a collection \((X_n)_{n\in\mathbb N}\) of \(\mathcal X\) valued random variables such that
\begin{equation}\label{MC}
\mathbb P\big(X_0\in A_0, \dots, X_n\in A_n\big) = \int\limits_{A_0} \gamma(\mathrm{d}x_0) \int\limits_{A_1} K(x_0, \mathrm{d}x_1) \cdots \int\limits_{A_n} K(x_{n-1}, \mathrm{d}x_n)
\end{equation}
for all \(A_1, \dots, A_n\in\mathcal B(\mathcal X)\) where \(\gamma\) denotes the distribution of \(X_0\).
\end{enumerate}
\end{defi}

We will call \(\gamma\) the \emph{initial} or \emph{starting distribution} of the Markov chain and will denote the distribution of this Markov chain by \(\mathbb P_\gamma\) and the expectation with respect to it by \(\mathbb E_\gamma[\cdot]\). Further an easy application of Kolmogorov�s consistency theorem implies that there is a measure \(\mathbb P_\gamma\) on the \emph{path space} \(\mathcal X^{\mathbb N}\) that satisfies \eqref{MC} which shows the existence of a Markov chain given a transition kernel \(K\) and initial distribution \(\gamma\) (cf. \cite{le2016brownian}). If the initial distribution is deterministic, i.e. \(\gamma = \delta_x\) for one \(x\in\mathcal X\), then we also write \(\mathbb P_x\) for the distribution of the Markov chain.
We close this paragraph by introducing the notation
\[K^n(x, A) \coloneqq \mathbb P_x(X_n\in A) \]
which is consistent with \eqref{MC} for \(n=1\).

\subsubsection*{Irreducibility, recurrence and existence of stationary distributions}

From now on we will fix a reference measure \(\mu\) on \(\mathcal X\).

\begin{defi}[Irreducibility and recurrence]
\begin{enumerate}
\item We say a Markov chain is \emph{\(\mu\) irreducible} if for every \(A\in\mathcal B(\mathcal X)\) with \(\mu(A)>0\) there is an index \(n\in\mathbb N\) such that
\[\mathbb P_x(X_n\in A) = K^n(x, A) > 0 \quad \text{for all } x\in\mathcal X. \]
\item A Markov chain \((X_n)_{n \in\mathbb N}\) is called \emph{recurrent} if
\begin{enumerate}
\item there is a measure \(\mu\) on \(\mathcal B(\mathcal X)\) such that \((X_n)\) is \(\mu\)-irreducible and
\item for every \(A\in\mathcal B(\mathcal X)\) with \(\mu(A)>0\) the expected number of visits of \(A\) is infinite, i.e.
\[\mathbb E_x\left[\left\lvert \left\{ n\in\mathbb N \mid X_n\in A\right\} \right\rvert\right] = \infty \quad \text{for every } x\in A.\]
\end{enumerate}
\item A Markov chain is called \emph{Harris recurrent} if it is recurrent and the number of visits is almost surely infinite, i.e. for any \(A\in\mathcal B(\mathcal X)\) with \(\mu(A)>0\) we have
\[\mathbb P_x\left(\left\lvert \left\{ n\in\mathbb N \mid X_n\in A\right\} \right\rvert = \infty\right) = 1 \quad \text{for every } x\in A.\]
\end{enumerate}
\end{defi}

\begin{defi}[Stationary distribution]
Let \(\pi\) be a measure on \(\mathcal B(\mathcal X)\). We call \(\pi\) an \emph{invariant} or \emph{stationary distribution} of a Markov chain with kernel \(K\), if \(X_{n+1}\) is distributed according to \(\pi\) whenever \(X_n\) is distributed according to \(\pi\). This is equivalent to
\[\pi(A) = \int\limits_{\mathcal X} K(x, A)\pi(\mathrm d x) \quad \text{for all } A\in\mathcal B(\mathcal X). \]
\end{defi}

\begin{theo}[Existence of stationary distributions]
If \((X_n)_{n\in\mathbb N}\) is a recurrent Markov chain, there exists an invariant \(\sigma\)-finite measure which is unique up to a multiplicative factor.
\end{theo}

\subsubsection*{Convergence to the stationary distribution and ergodicity}

We will not introduce the notion of periodic and aperiodic Markov chains here, because it would distract us from our actual goal. However, we still present the following result that only holds for aperiodic Markov chains and refer to \cite{meyn2012markov} for further information. The reason why we present the theorem is that it explains how one can approximately sample from the stationary distribution of a Markov chain, namely it says that the distribution of \(X_n\) converges to the invariant distribution.

\begin{theo}[Convergence to stationary distribution]
Let \((X_n)_{n\in\mathbb N}\) be a Harris recurrent and aperiodic Markov chain with stationary distribution \(\pi\). Let further \(\gamma_n\) be the distribution of \(X_n\), then we have
\[\left\lVert \gamma_n - \pi \right\rVert_{TV} \xlongrightarrow{n\to\infty} 0 \]
non increasing. Here \(\left\lVert \cdot \right\rVert_{TV}\) denotes the total variation of a measure
\[\left\lVert \mu \right\rVert_{TV}\coloneqq \sup_{\mathcal E}\sum\limits_{E\in\mathcal E} \left\lvert \mu(E) \right\rvert\]
where the supremum is taken over all finite families of disjoint measurable sets.
\end{theo}

\begin{theo}[Ergodic theorem]\label{ergTheo}
Let \((X_n)_{n\in\mathbb N}\) be a Harris recurrent Markov chain with stationary probability distribution \(\pi\), then \((X_n)_{n\in\mathbb N}\) is \emph{ergodic}. This means that if 
\[\hat{\mathbb P}_n \coloneqq \frac1n \sum_{i=1}^n \delta_{X_i} \]
is the empirical measure, we almost surely have
\begin{equation}\label{ergo}
\int\limits_{\mathcal X} f(x) \hat{\mathbb P}_n(\mathrm dx) \xlongrightarrow{n\to\infty} \int\limits_{\mathcal X} f(x)\pi(\mathrm dx)
\end{equation}
for every \(\pi\) integrable function \(f\).
\end{theo}

In the particular case that \(\mathcal X\) is a topological space and \(\mathcal B(\mathcal X)\) is the Borel algebra and if \(\pi\) is a probability measure, we obtain the almost sure weak convergence of \(\hat{\mathbb P}_n\) towards \(\pi\). This means that the convergence in \eqref{ergo} almost surely holds for all continuous and bounded functions \(f\). Hence, \(\hat{\mathbb P}_n\) are approximations of the invariant distribution in the sense of weak convergence, which is metrisable for example by the L�vy-Prokhorov or the bounded dual Lipschitz metric (cf. \cite{dudley2010distances}).


\subsubsection*{Idea of Markov chain Monte Carlo methods}

The motivation of the study of Markov chain Monte Carlo methods was to approximate the posterior distribution \eqref{post2}. The idea is now to construct and then simulate a Markov chain \((X_n)_{n\in\mathbb N}\) such that the empirical measures \(\hat{\mathbb P}_n\) converge to the posterior.
\begin{defi}[MCMC methods]
A \emph{Markov chain Monte Carlo} (\emph{MCMC}) method for the simulation of a distribution \(\pi\) is any method that produces an ergodic Markov chain \((X_n)_{n\in\mathbb N}\) with stationary distribution \(\pi\).
\end{defi}

In order to achieve this we only have to construct a suitable Markov chain and check the requirements of the ergodic theorem. This means we want to construct a Harris recurrent Markov chain with invariant distribution \(\pi\) and we want to do this without having to compute the normalisation constant \eqref{norma}. We will now present two of the most common methods to do this which are the Metropolis-Hastings random walk and the method of slice sampling.

\subsection{Metropolis-Hastings random walk}

The Metropolis-Hastings random walk is arguably the most commonly used MCMC method and certainly one of the oldest. It was actually proposed in the early 1950s from researchers of the American nuclear programme in Los Alamos (cf. \cite{metropolis1953equation}). First we will touch on the theoretical aspects of this method and follow the presentation in \cite{robert2013monte}.

\enlargethispage{.2cm}
\begin{emp}[Setting]
Let \(\Theta\) be a measurable space, \(\mu\) a measure on that space and \(f\colon\mathcal X\to[0, \infty]\) a function with finite positive integral
\[Z \coloneqq \int\limits_{\mathcal X} f(x) \mu(\mathrm{d}x)\in(0, \infty).\]
Our goal is to find a Harris recurrent Markov chain with invariant distribution
\[\pi(A)\coloneqq \frac1Z \int\limits_A f(x)\mu(\mathrm dx). \]
Let further
\[\Big\{ f(\cdot|x) \;\big\lvert\; x \in\mathcal X\Big\}\]
be a family of probability distributions, which we call the \emph{proposal distributions}.
\end{emp}

\begin{emp}[The MH random walk]
Given the first states \(X_0 = x_0, \dots, X_n = x_n\) of the Markov, we define \(X_{n+1}\) as follows. Let \(Y\) be distributed according to \(f(\cdot|x_n)\mathrm d\mu\) and take one realisation \(y\) of \(Y\). Then set
\[X_{n+1}\coloneqq\begin{cases}\; y \quad & \text{with probability } \rho(x_n, y) \\\; x_n & \text{with probability } 1 - \rho(x_n, y) \end{cases} \]
where
\begin{equation}\label{threshold}
\rho(x, y) \coloneqq \min\left\{ \frac{f(y) f(x|y)}{f(x)f(y|x)}, 1\right\}.
\end{equation}
and \(\frac{a}{0} \coloneqq \infty\). The first step of the random walk, namely the sampling of \(y\) is called the \emph{proposal step} and the second one the \emph{accept-reject step}. In conclusion a single step of the MH random walk can be expressed in the following way. 
\begin{algorithm}
\caption{A single step of the MH random walk \label{alg:MH}}
\begin{algorithmic}[1]
\Require{Current state \(x_n\) of the MH random walk}
\State \(y\sim f(\cdot|x_n)\mathrm d\mu\)
\State \(a\sim \mathcal U([0, 1])\)
\If{\(a\le \rho(x_n, y)\)}
  \State \(x_{n+1}\gets y\)
\Else 
  \State \(x_{n+1}\gets x_n\)
\EndIf
\State\Return{\(x_{n+1}\)}
\end{algorithmic}
\end{algorithm}
\end{emp}

To see that the definition above indeed yields a Markov chain we convince ourselves that the transition kernel is given by
\[K(x, A) = \int\limits_A\rho(x, y)f(y|x) \mu(\mathrm dy) + (1 - m(x)) \delta_x(A)\]
where \(\delta_x\) is the Dirac measure in \(x\) and
\[m(x) = \int\limits_{\mathcal X} \rho(x, y) f(y|x) \mu(\mathrm dy)\in[0, 1] \]
is the \emph{acceptance probability} of the chain at state \(x\).

\begin{prop}[Stationary distribution]
The probability measure \(\pi\) is a stationary distribution of the MH random walk.
\end{prop}
\begin{proof}
We have
\begin{equation}\label{calc1}
\begin{split}
\int\limits_{\mathcal X} K(x, A) \pi(\mathrm dx) & =  \frac1Z \int\limits_{\mathcal X}\left( \int\limits_A \rho(x, y) f(y|x) \mu(\mathrm dy) + (1 - m(x)) \delta_x(A)\right) f(x) \mu(\mathrm dx) 
\end{split}
\end{equation}
We note that
\[\rho(x, y) f(y|x) f(x) = \rho(y, x) f(x|y) f(y).\]
Furthermore we can compute
\begin{equation*}
\begin{split}
 \int\limits_{\mathcal X} m(x) \delta_x(A) f(x) \mu(\mathrm dx) & = \int\limits_{A}\int\limits_{\mathcal X} \rho(x, y) f(y|x) \mu(\mathrm dy) f(x) \mu(\mathrm dx) \\
 & =\;\int\limits_{\mathcal X} \int\limits_A \rho(x, y) f(y|x)f(x) \mu(\mathrm dx) \mu(\mathrm dy) \\
 & =\;\int\limits_{\mathcal X} \int\limits_A \rho(y, x) f(x|y) \mu(\mathrm dx) f(y) \mu(\mathrm dy)
\end{split}
\end{equation*}
where we used Fubini-Tonelli theorem\footnote{The Fubini-Tonelli theorem states that the order of integration with respect to two \(\sigma\)-additive measures can be swapped, if the integrated function is non negative.} in the second to last step. We note that two of the terms in \eqref{calc1} cancel out and we obtain
\[\int\limits_{\mathcal X} K(x, A) \pi(\mathrm dx) = \frac1Z \int\limits_{\mathcal X} \delta_x(A) f(x) \mu(\mathrm dx) = \pi(A). \]
\end{proof}

Now we aim to prove that the MH random walk is Harris recurrent because then the ergodic theorem yields that the empirical measures associated with the Markov chain will actually converge to \(\pi\). Obviously this is not trues for all proposal families in general the case, for example we could consider the case where the proposal distribution \(f(\cdot|x)\) is just the Dirac measure in \(x\).\footnote{Obviously this is slightly formal, because the Dirac measure can typically not be expressed through a density. However, rigorous examples can be constructed similarly.} Then the MH random walk would never leave its initial position which will typically be a deterministic point. Hence, the empirical measures are only the Dirac measure in the starting point and will not converge towards \(\pi\).

The first step towards Harris recurrence is to show irreducibility and this will already give us some hints to what families of proposal are sensible.

\begin{prop}[Irreducibility]
Assume that the proposal family is strictly positive, i.e.
\[f(y|x) > 0 \quad \text{for all } x, y\in\mathcal X. \]
Then the MH random walk is \(\pi\) irreducible.
\end{prop}
\begin{proof}
For any measurable set \(A\subseteq\mathcal X\) with positive measure \(\pi(A)>0\) we have
\[K(x, A) \ge \int\limits_A \rho(x, y)f(y|x) \mu(\mathrm d y) > 0. \]
To see this, we can assume that this would not hold and then the integrant has to be zero \(\mu\) almost surely. Since \(f(y|x)\) is strictly positive this would imply \(\rho(x, y) = 0\) and hence \(f(y) = 0\) for \(\mu\) almost all \(y\in A\). However, this is a contradiction to
\[\pi(A) = \int\limits_A f(y)\mu(\mathrm dy) > 0.\]
\end{proof}

Now we can formulate the ergodicity for \(\pi\) irreducible MH random walks.

\begin{theo}[Ergodicity of the MH random walk]
If the MH random walk is \(\pi\) irreducible, then it is also Harris recurrent and hence ergodic.
\end{theo}
\begin{proof}
We refer to Lemma 7.3 in \cite{robert2013monte} for the proof of Harris recurrency, the ergodicity then follows from the ergodic theorem.
\end{proof}

\subsubsection*{Implementation of the MH random walk}

So far we have presented the theoretical foundations of the MH random walk and now we want to touch on a few aspect of the simulation of this Markov chain. For this part we shall point the reader towards the example based introductions \cite{robert1999metropolis} and \cite{robert2010introducing} to the implementation of the MH random walk which also provide coding examples. We have seen that the empirical measures associated with the MH random walk converge to \(\pi\) under fairly mild assumptions, meaning for a wide class of proposal distributions. Nevertheless it is mostly the choice of the proposal that determines the speed of this convergence. 
 In order to shortly demonstrate this effect, we consider the case \(\mathcal X = \mathbb R^d\) and that the reference measure \(\mu\) is the Lebesgue measure.

\begin{emp}[Choosing a proposal family]
Usually one chooses the proposal such that the expectation of \(f(\cdot|x)\) is \(x\). The most common choice of a proposals is a family of normal distributions \(f(\cdot|x)\) with expectation \(x\) and covariance \(\Sigma\in\mathbb R^{d\times d}\). This also has the effect that the acceptance ratio takes the easier form
\[\rho(x, y) = \min\left\{ \frac{f(y)}{f(x)}, 1\right\}.\]
Also since the densities are strictly positive we ensure that the resulting Markov chain is \(\pi\) irreducible.
\end{emp}

\begin{emp}[Acceptance rate, autocorrelation and effective sample size]
Once we have agreed to stick to normal densities for the proposal distributions, we still have the freedom to choose the covariance \(\Sigma\in\mathbb R^{d\times d}\). This determines how far the proposed new values will be away from the current state of the Markov chain. The motivation for an aggressive proposal distribution, i.e. for a high variance would be that this would enable the Markov chain to take bigger steps and hence explore the space \(\mathcal X\) faster. Also the chain would be more likely to jump between possibly isolated areas of high density. However, this could also lead to a high rejection rate\footnote{The term should be rather intuitive; the rejection rate is the relative amount of rejections that occurred in the MH random walk and analogously the acceptance rate is its counterpart.} if the proposed values are often so far away from the current state of the Markov chain that they are lie within an area of low density. In this case the Markov chain will only �visit� few distinct points in the space \(\mathcal X\) which is also very unfavourable. In fact the findings in \cite{roberts1997weak} suggests that an acceptance rate around \(25\%\) is desirable in dimension \(d\ge3\) and around \(50\%\) for dimension \(d=1, 2\). The connection between the proposal distribution and the acceptance rate is also elaborated in the upcoming example.

The \emph{autocorrelation function} (\(\operatorname{acf}\)) of a sequence of data points \(x_0, \dots, x_n\) captures the estimated correlation between the observations. More precisely \(\operatorname{acf}(k)\) gives the empirical correlation\footnote{This is the correlation of the two empirical measures associated with \((x_0, \dots, x_{n-k})\) and \((x_k, \dots, x_n)\).} of \((x_0, x_1, \dots, x_{n-k})\) and \((x_k, x_{k+1}, \dots, x_n)\). In the case that the data points are generated by a MH random walk, the autocorrelation function determines the correlation of the Markov chain at time \(l\) with the Markov chain at time \(l+k\). Hence, if \(\operatorname{acf}(k)<\varepsilon_0\) where \(\varepsilon_0>0\) is fixed in advance, one can perceive \(x_0, x_k, x_{2k}, \dots\) as an independent sequence of samples from \(\pi\)  -- or more precisely an only weakly correlated one. The \emph{effective sample size} is the length \(m\) of this new almost uncorrelated sequence \(x_0, x_k, x_{2k}, \dots, x_{mk}\). Obviously the effective sample size strongly depends on the choice of \(\varepsilon_0\) that incorporates how much correlation one is willing to accept.

We should quickly touch on how the proposal affects the autocorrelation function and hence the effective sample size. Assume we have a very aggressive proposal distribution. Then we will typically have a high rejection rate and hence \(x_l = x_{l+k}\) a lot of times meaning that the autocorrelation function will be high and therefore the effective sample size is rather low. On the other hand if the proposal is too conservative the MH random walk will only take very small steps and hence \(x_{l+k}\) will still be close to \(x_l\). Therefore, the autocorrelation will be high and the effective sample size low again. This effect of the proposal can be seen in Figure \ref{fig:4.1.2}.
\end{emp}

\begin{ex}[One dimensional MH]\label{example}
We follow an examples for a one dimensional MH random walk given in \cite{robert1999metropolis} and we want to approximate the probability distribution with unnormalised density
\[f(x)\coloneqq \sin(x)^2\cdot\sin(2x)^2\cdot\exp\left(-\frac{x^2}{2}\right). \]
The goal of this example is to see how different proposal distributions lead to different acceptance rates, a different exploration of the state space \(\mathcal X = \mathbb R\) and different effective sample sizes. In order to achieve this, we run \(2\cdot10^4\) samples of the MH random walk with starting point \(x_0=1\) and three different values \(\alpha = 0.01, 3, 100\) for the variance of the proposal distributions. Then we plot a histogram including the actual density and the autocorrelation function for all different values. The acceptance rates where approximately \(88\%\) for \(\alpha=0.01\), \(34\%\) for \(\alpha=3\) and \(9\%\) for \(\alpha=100\). The effective sample sizes for the three different chains different values for \(\alpha\) 
are
\[\frac{2\cdot10^4}{50} = 4\cdot10^2, \quad \frac{2\cdot10^4}{8} = 2.5\cdot10^3 \quad \text{and } \frac{2\cdot10^4}{30} \approx 7\cdot10^2 \]
for \(\alpha=0.01, 3, 100\) respectively.

This simulation illustrates the problem of too aggressive -- \(\alpha = 100\) -- and too conservative -- \(\alpha=0.01\) -- proposal distributions and shows how this effects the acceptance rate and the effective sample size.

\begin{figure}[h!]
	\centering
	\includegraphics[width=0.41\textwidth]{figures/alpha-small}
	\includegraphics[width=0.41\textwidth]{figures/auto-alpha-small}
	\includegraphics[width=0.41\textwidth]{figures/alpha-optimal}
	\includegraphics[width=0.41\textwidth]{figures/auto-alpha-optimal}
	\includegraphics[width=0.41\textwidth]{figures/alpha-big}
	\includegraphics[width=0.41\textwidth]{figures/auto-alpha-big}
	\caption{Histograms and autocorrelation functions of for three different variances \(\alpha\) of the Gaussian proposal distributions. It is apparent that the histogram for \(\alpha=3\) fits the actual density the best and also that the autocorrelation decays the quickest for this parameter. Note that for \(\alpha=0.01\) the MH random walk only explored some area of high density. The actual density is obtained by numerical integration.}
	\label{fig:4.1.2}
\end{figure}

\end{ex}

\begin{emp}[Tuning the proposal]\label{tuning}
In order to obtain a higher acceptance rate without simply choosing the variance of the proposal distribution small one can \emph{tune} or \emph{adapt} the proposal distribution. This means one adjusts the proposal distribution after a while, lets say after the first \(10^3\) samples in such a way that one replaces the original covariance matrix \(\Sigma\) by the empirical covariance of the first \(10^3\) samples. Then one forgets about all the samples so far -- they are usually called the \emph{burn in period} -- and starts a new MH random walk. It is essential to drop the first samples since otherwise the Markov property would break as all further samples now rely on the coveriance of the first burn in period and hence on those points. 
The reason why this increases the acceptance rate is, that the proposal now only is aggressive in those directions where the mass of the density is widely spread. For a further discussion we refer to \cite{roberts2009examples}.
\end{emp}

\begin{emp}[The Gelman-Rubin diagnostic]
So far we have seen guidelines as what properties of the MCMC simulation can be seen as favourable or not. However those comments can not replace quantitative measures on the convergence of the simulated Markov chains and one of them is the Gelman-Rubin diagnostics which is also called the \(\hat R\) value. We will not be able to rigorously introduce this quantity, but will make a few comments since we will use it later and refer to \cite{robert2012discretization} for a thorough introduction to convergence diagnostics for MCMC methods and to \cite{gelman1992inference} and \cite{brooks1998general} for the orinigal work by Gelman, Rubin and Brooks. In a nutshell the \(\hat R\) value is an estimate of how much longer a MCMC simulation would have to run to be a good approximation of the stationary distribution. It is generally accepted that a \(\hat R\) value of at most \(1.05\) can be taken as a sign -- but not a proof -- of convergence, cf. \cite{brooks1998general}.

Although we don�t introduce the statistics itself, we shall present the requirements to compute it. The procedure one has to take is the following:
\begin{enumerate}
\item Find the possibly multiple modes of the distribution that should be approximated. This can be done either by exploiting optimisation algorithms or running short MCMC simulations, which we will do later in our toy example.
\item Run \(m\) MCMC simulations of length \(n\) starting at random points with variance greater than the estimated variance of the target distribution \(\pi\). This variance is typically estimated through a first, shorter MCMC simulation which can also be used to tune the proposal.
\end{enumerate}
Now the \(\hat R\) value can be computed from the entirety of those \(m\) chains of length \(n\) and we will rely on a pre-implemented tool in R to do this.
\end{emp}

\subsection{Slice sampling}

Slice sampling is a different MCMC method and quite similar to the MH random walk. Nevertheless it has the benefit that one does not have to define or tune a family of proposal distributions and that the constructed Markov chain is always irreducible. However, we will see that at least when one wants to simulate the slice sampling one runs into similar problems of having to choose a parameter that influences the auto correlation function and hence the speed of convergence of the method. We begin by fixing our frame we will work in.

\begin{emp}[Setting]
Let \(\mathcal X\) be a measurable space, \(\mu\) a measure on that space and \(f\colon\mathcal X\to[0, \infty]\) a function with finite integral
\[Z \coloneqq \int\limits_{\mathcal X} f(x) \mu(\mathrm{d}x)\in(0, \infty).\]
In particular there is \(\hat x\in\mathcal X\) such that \(f(\hat x)>0\). Our goal is to find an ergodic Markov chain with invariant distribution
\[\pi(A)\coloneqq \frac1Z \int\limits_A f(x)\mu(\mathrm dx). \]
Further we will assume -- after an eventual modification of \(f\) on a \(\mu\) Null set -- that 
\[f\le \left\lVert f \right\rVert_{L^\infty(\mu)} = \inf \Big\{ \alpha\in\mathbb R \;\big\lvert\; f \le \alpha \text{ almost surely with respect to } \mu \Big\}\in[0, \infty]. \]
\end{emp}

\begin{emp}[The slice sampling method]
Assume we have already given the first \(n\) samples \(x_1, \dots, x_n\) of the Markov chain. If we have \(f(x_n)=0\), then we set \(x_{n+1}\coloneqq \hat x\). Otherwise we sample \(y\) according to the uniform distribution on \([0, f(x_n)]\) and define the \emph{slice}
\[S\coloneqq S(y)\coloneqq \left\{ x\in\mathcal X \mid f(x) \ge y \right\}. \]
\begin{figure}[h!]
	\centering
	\includegraphics[width=0.6\textwidth]{figures/Slice-sampling-neal-new}
	\caption{Schematic sketch of the selection of a slice: (a) first \(y\) is sampled uniformly in \([0, f(x_0)]\) and (b) the slice is selected. Original graphic from \cite{neal2003slice}.}
	\label{fig:4.2}
\end{figure}

Note that because \(y < f(x_n)\le \left\lVert f \right\rVert_{L^\infty(\mu)}\) holds almost surely, we have \(\mu(S)>0\) as well as
\[ \mu(S) \le y^{-1} \int\limits_S f(x)\mu(\mathrm dx)<\infty \]
where we used Markov�s inequality as well as \(y>0\) almost surely. Now draw \(x_{n+1}\) according to the uniform distribution\footnote{Of course we mean the uniform distribution with respect to \(\mu\) that gives weight \(\mu(S)^{-1} \cdot \mu(A)\) to a set \(A\subseteq S\).} on \(S\). Note that if \(f(x_n)>0\), then \(f(x_{n+1})\ge y > 0\) almost surely, hence \(f(x_n) = 0\) can only hold for \(n=0\). Further the reason why we have to treat the case \(f(x_n)=0\) individually is, that there typically is no uniform distribution on the slice \(S(0) = \mathcal X\).
The pseudo code for the construction of the resulting Markov chain is presented in Algorithm \ref{alg:slice-sampling}.
\begin{algorithm}
\caption{A single slice sampling step \label{alg:slice-sampling}}
\begin{algorithmic}[1]
\Require{Current state \(x_n\) of the Markov chain}
 \If \(f(x_n) = 0\)
  \State \(x_{n+1}\gets \hat x\)
\Else
  \State \(y\sim \mathcal U([0, f(x_n)])\)
  \State \(S\gets \left\{ x\in\mathcal X \mid f(x) \ge y \right\}\)
  \State \(x_{n+1} \sim \mathcal U(S)\)
\EndIf
\State\Return{\(x_{n+1}\)}
\end{algorithmic}
\end{algorithm}
\end{emp}

If we compare the Markov chain to the MH random walk, we notice that in the slice sampling we first create a random threshold \(y\) and then sample uniformly from all points that satisfy this threshold. This is just the other way round than in the MH random walk where we first make a proposal for the next state of the Markov chain and then decide whether we will accept it or not.

Just like in the case of the MH random walk we can explicitly give the transition kernel and use this expression then to check that \(\pi\) is a stationary distribution. The kernel of the Markov chain that arises from the slice sampling iteration is given by
\begin{equation*}
\begin{split}
K(x, A) & = \int\limits_\mathbb R\frac{\mathds{1}_{[0, f(x)]}(y)}{f(x)} \cdot \frac{\mu(A\cap S(y))}{\mu(S(y))} \lambda(\mathrm dy) \\
& = \int\limits_\mathbb R\frac{\mathds{1}_{[0, f(x)]}(y)}{f(x)} \cdot \frac1{Z(y)}\int\limits_A \mathds{1}_{[y, \infty)}(f(z)) \mu(\mathrm dz) \lambda(\mathrm dy)
\end{split}
\end{equation*}
where \(\lambda\) is the Lebesgue measure on \(\mathbb R\), \(\mathds{1}\) is the indicator function and \(Z(y)\) is the normalisation constant
\[Z(y)\coloneqq \int\limits_{\mathcal X} \mathds{1}_{[y, \infty)}(f(z)) \mu(\mathrm dz) = \mu(S(y)) \in(0, \infty). \]
Obviously the expression above only holds if \(f(x)>0\) and in the case \(f(x) = 0\) we have
\[K(x, A) = \delta_{\hat x}(A).\]

\begin{prop}[Invariant distribution]
The probability distribution \(\pi\) is a stationary distribution of the Markov chain associated with the slice sampling method.
\end{prop}
\begin{proof}
For any \(A\subseteq\mathcal X\) we can compute
\begin{equation*}
\begin{split}
\int\limits_{\mathcal X} K(x, A) \pi(\mathrm dx) & = \frac1Z\int\limits_{\mathcal X }\int\limits_\mathbb R\frac{\mathds{1}_{[0, f(x)]}(y)}{f(x)}\cdot \frac1{Z(y)} \int\limits_A \mathds{1}_{[y, \infty)}(f(z)) \mu(\mathrm dz) \lambda(\mathrm dy) f(x) \mu(\mathrm dx) \\
& =\frac1Z \int\limits_{ A }\int\limits_\mathbb R \frac1{Z(y)} \int\limits_{\mathcal X}\mathds{1}_{[y, \infty)}(f(x)) \mu(\mathrm dx) \mathds{1}_{[0, f(z)]}(y) \lambda(\mathrm dy) \mu(\mathrm dz) \\
& = \frac1Z\int\limits_A f(z)\mu(\mathrm dz) = \pi(A)
\end{split}
\end{equation*}
where we again used Fubini�s theorem for non negative functions.
\end{proof}

\begin{prop}[Irreducibility]
The Markov chain that arises from the slice sampling algorithm is \(\pi\) irreducible.
\end{prop}
\begin{proof}
Fix \(A\subseteq\mathcal X\) with positive probability \(\pi(A)>0\) and \(x\in\mathcal X\). If we have \(f(x)>0\), then we have \(\mu(A\cap S(y))>0\) for one \(y\in(0, f(x))\). We obtain
\[K(x, A) \ge \int\limits_{\mathbb R} \frac{\mathds{1}_{[0, y]}(z)}{f(x)} \cdot \frac{\mu(A\cap S(z))}{\mu(S(z))}> 0. \]
If however \(f(x)=0\), then we get
\[K^2(x, A) = K(\hat x, A) > 0. \]
\end{proof}

\begin{theo}[Ergodicity]
If \(f\) is bounded, the Markov chain induced by the slice sampling method is ergodic.
\end{theo}
\begin{proof}
See Theorem 6 in \cite{mira2002efficiency}.
\end{proof}

\subsubsection*{Implementation details}

Just like in the case of the MH random walk we will make a few comments about the actual simulation of the slice sampling algorithm and for this, we will assume \(\mathcal X\subseteq\mathbb R^d\).

The main difficulty in the implementation is the sampling of a uniform distribution on a slice \(S\). In practice it is not even possible to calculate the slice but one can exploit the following observation. Assume that we are able to simulate a uniform distribution on a set \(C\) that contains the slice \(S\). Then the following algorithm -- which is nothing but the conditioning of this uniform distribution on the event that the outcome is in \(S\) -- samples uniformly from \(S\).
\begin{algorithm}
\caption{Sampling from a uniform distribution on a subset \(S\subseteq C\) \label{alg:slice-sample}}
\begin{algorithmic}[1]
\Require{\(S\) and \(C\supseteq S\)}
\State \(x\sim\mathcal U(C)\)
\While {\(x\notin S\)}
  \State \(x\sim\mathcal U(C)\)
\EndWhile
\State\Return{\(x\)}
\end{algorithmic}
\end{algorithm}

An obvious choice for \(C\) would be a cuboid 
\[C = \prod_{i = 1}^d [a_i, b_i]\]
since it is straight forward to sample from a uniform distribution on a cuboid. Namely one only has to sample the individual coordinates uniformly in the intervals \([a_i, b_i]\). The problem still remains how one can find a cuboid that surely contains the whole slice \(S\). The short answer is that there is no general way to do this. However, not everything is lost, since we can use random cuboids that have the property that every part of the slice is contained in the cuboid with positive probability. This will be crucial in retaining the irreducibility of the Markov chain. In fact it has been found that in applications the following procedure works well (cf. \cite{affandi2014learning}). Given the current state \(x_n\) of the Markov chain, we propose a random interval \([a_i, b_i]\) around the \(i\)-th component of \(x_n\). Then we extend those intervals until the endpoints \(a\) and \(b\) of the cuboid do not lie in the slice anymore which is described in Algorithm \ref{alg:cuboid}.
\begin{algorithm}
\caption{Sampling a random cuboid \label{alg:cuboid}}
\begin{algorithmic}[1]
\Require{Current state \(x_n\) of the Markov chain, parameter \(\alpha>0\)}
\For {\(i=1, \dots, d\)}
  \State \(a_i, b_i \sim \mathcal E(\alpha)\)
\EndFor
\State \(a\gets (a_1, \dots, a_d), b\gets(b_1, \dots, b_d)\)
\While {\(x - a \in S\)}
  \State \(a \gets 2 \cdot a\)
\EndWhile
\While {\(x + b \in S\)}
  \State \(b \gets 2 \cdot b\)
\EndWhile
\State\Return{\((x - a, x + b)\)}
\end{algorithmic}
\end{algorithm}

Here \(\mathcal E(\alpha)\) denotes the exponential distribution with parameter \(\alpha\) and determines how large the first proposed intervals are. It is straight forward and computationally very easy to determine whether a point \(x\) is in the slice \(S(y)\) since one only has to check \(f(x) \ge y\). The reason for the choice of the exponential distribution is that this ensures that the cuboid can get arbitrarily large with positive probability. This leads to the effect that the Markov chain one obtains in exchanging the sample from \(\mathcal U(S)\) by a sample from \(\mathcal U(S\cap C)\) still is irreducible. To see this we can slightly modify the proof of irreducibility, so for \(A\subseteq\mathcal X\) with positive probability we choose \(y>0\) such that \(\mu(A\cap S(y))>0\). Further we can choose a cuboid \(C\) around \(x\) such that \(\mu\big(A\cap S(y)\cap C\big)>0\). Further this cuboid is contained in the cuboid proposed by Algorithm \ref{alg:cuboid} with positive probability and hence we have
\[K(x, A) = \mathbb P_x(X_1\in A) > 0. \]
The algorithm that arises from the combination of the usual slice sampling method and the approximation of the uniform distribution on the slice is presented in Algorithm \ref{alg:slice-sampling-implementation}.
\begin{algorithm}
\caption{Algorithm for the slice sampling \label{alg:slice-sampling-implementation}}
\begin{algorithmic}[1]
\Require{Unnormalised density \(f\), starting value \(x_0\), desired length \(n\) of the chain, \(\alpha>0\)}
\If {\(f(x_0) = 0\)}
  \State \(x_0 \gets \hat x\)
\EndIf
\For {\(i = 0, \dots, n-1\)}
  \State \(y\sim \mathcal U([0, f(x_{i})])\)
  \State \(C\) random cuboid around \(x_{i}\) with parameter \(\alpha\)
  \State \(x\sim\mathcal U(C)\)
  \While {\(f(x)<y\)}
    \State \(x\sim\mathcal U(C)\)
  \EndWhile
  \State \(x_{i+1} \gets x\)
\EndFor
\State\Return{\(x = (x_0, \dots, x_n)\)}
\end{algorithmic}
\end{algorithm}
It shall be noted, that this algorithm also uses a point \(\hat x\) of positive density, which can be determined easily for a lot of densities \(f\). If this is not straight forward, one can sample \(x_0\) according to a normal distribution until we select a point of positive density.

Obviously the algorithm presented above produces a Markov chain that is not identical with the one presented in the theoretical discussion of the slice sampling method. However, if one wants to ensure the convergence of this slightly modified Markov chain, one has to check whether \(\pi\) remains a stationary distribution and whether the chain is still ergodic. This is usually done in the specific setting one works in, cf. \cite{neal2003slice}. We will quickly discuss this in a very easy case. Namely let us assume \(d=1\) and that \(f\) is continuous and has only one local maximum. We call \(f\) \emph{unimodal} in this case and note that every slice \(S(y)\) is an interval. Hence, the proposed cuboid is an interval around \(x_n\in S(y)\) such that both endpoints are outside of the slice \(S(y)\) and hence we have \(S(y)\subseteq C\). Therefore, the algorithm above is equivalent to the original slice sampling method and hence produces an ergodic Markov chain with the desired invariant distribution.

\begin{emp}[The choice of \(\alpha\)]
One could think that a small choice of \(\alpha\) -- which relates into large values of \(a_i\) and \(b_i\) -- would be the best since this increases the probability that the whole slice \(S\) is contained in the cuboid \(C\). There is some truth in this approach, since \(\mathcal U(S\cap C)\) is a better approximation of \(\mathcal U(S)\) if \(C\) is larger and further the while loops in Algorithm \ref{alg:cuboid} need less repetitions if \(a_i\) and \(b_i\) initially are big. However, one should not choose \(\alpha\) too small, because a large cuboid \(C\) also means that a lot of samples from \(\mathcal U(C)\) will lie outside of \(S\cap C\). Hence, Algorithm \ref{alg:slice-sample} that samples from \(\mathcal U(S\cap C)\) will get slower as it will reject a lot of samples.

In conclusion there is a trade off in terms of computation time between the choice of too small and too large values for \(\alpha\). However not always the parameter \(\alpha\) that minimises the simulation time is the most suitable, since the autocorrelation decreases together with the parameter \(\alpha\). Hence, computation time should rather be compared to the effective sample size.

Those effects of \(\alpha\) on the auto correlation and therefore effective sample size can be seen in Figure \ref{fig:4.3} where the procedure of Example \ref{example} is repeated but this time with the slice sampling method. The sample size remains \(2\cdot10^4\) and the different parameter choices where \(\alpha = 0.01, 0.5, 10\). The according computation times where approximately \(26\si{s}\) for \(\alpha=0.01\), \(1.7\si{s}\) for \(\alpha=0.5\) and \(1.7\si{s}\) for \(\alpha=10\). In regard of the decay of the autocorrelation functions and the resulting effective sample sizes, it is apparent that the choice \(\alpha = 0.5\) would be the most sensible one in this case.

\begin{figure}[h!]
	\centering
	\includegraphics[width=0.44\textwidth]{figures/slice-alpha-small}
	\includegraphics[width=0.44\textwidth]{figures/slice-auto-alpha-small}
	\includegraphics[width=0.44\textwidth]{figures/slice-alpha-optimal}
	\includegraphics[width=0.44\textwidth]{figures/slice-auto-alpha-optimal}
	\includegraphics[width=0.44\textwidth]{figures/slice-alpha-big}
	\includegraphics[width=0.44\textwidth]{figures/slice-auto-alpha-big}
	\caption{Histograms and autocorrelation functions for the choices of \(\alpha = 0.01, 0.5, 10\). The auto correlation obviously decreases the fastest for \(\alpha=0.01\), however the computation time is much higher than for the parameter the other parameters.}
	\label{fig:4.3}
\end{figure}
\end{emp}

\subsection{Variational MCMC methods}

Now that we have presented a general setup for MCMC methods we wish to use them to approximated the posterior distribution which is given by the unnormalised density 
\begin{equation}\label{post3}
f(\theta) = f_\Theta(\theta) \prod_{i=1}^n \frac{\det(L(\theta)_{Y_i})}{\det(L(\theta) + I)}.
\end{equation}
In the light of the theoretical guarantees this will surely work and actually Figure \ref{fig:4.1} has been created this way. However, the evaluation of this unnormalised density \(f\) can take several seconds or even minutes itself since it involves the computation of the determinant of the \(N\times N\) matrix \(L(\theta) + I\). However, one can efficiently compute bounds of the unnormalised density and we will provide a general setup of how the MH random walk and slice sampling can be expressed using those bounds. This will lead to significantly shorter simulation times for the respective MCMC methods.

\begin{emp}[Setting]
Let \(\mathcal X\) be a measurable space, \(\mu\) a measure on that space and \(f\colon\mathcal X\to[0, \infty]\) a function with finite positive integral
\[Z = \int\limits_{\mathcal X} f(x) \mu(\mathrm{d}x)\in(0, \infty).\]
Let further \(f\le \left\lVert f \right\rVert_{L^\infty(\mu)}\) and let
\[\left\{ f(\cdot|x) \mid x \in\mathcal X\right\}\]
be a family of proposal distributions.
Let now \(f_n^-, f_n^+\colon\mathcal X\to [0, \infty]\) be functions such that \(f_n^-(x)\le f(x)\le f_n^+(x)\) for all \(x\in\mathcal X\) as well as
\[f_n^\pm(x) \xlongrightarrow{n\to\infty} f(x) \quad \text{for all } x\in\mathcal X. \]
We seek an expression of the MH random walk and the slice sampling method that purely relies on those bounds \(f_n^-\) and \(f_n^+\) of the unnormalised density.
\end{emp}

\subsubsection*{Variational MH random walk}

We note that the only part in the algorithm for the MH random walk where \(f\) is needed itself is the accept-reject step, hence, it suffices to adjust this step. In order to achieve this we bound the acceptance rate through
\[\rho_n^\pm(x, y) \coloneqq \min\left\{ \frac{f_n^\pm(y) f(x|y)}{f_n^\mp(x)f(y|x)}, 1\right\}.\]
In fact we obviously have \(\rho_n^-(x, y)\le\rho(x, y)\le\rho_n^+(x, y)\) as well as
\[\rho_n^\pm(x, y)\xlongrightarrow{n\to\infty}\rho(x, y) \quad\text{for all } x, y\in\mathcal X. \]
Hence if we want to decide whether a number \(a\) satisfies \(a\le \rho(x, y)\) we can iteratively tighten the upper and lower bounds on \(\rho\) until we either have \(a\le\rho_n^-(x, y)\) and thus \(a\le\rho(x, y)\) or \(a>\rho_n^+(x, y)\) and therefore \(a>\rho(x, y)\).
Now we can adjust the algorithm of the MH random walk accordingly and obtain Algorithm \ref{alg:variational-MH}.

\begin{algorithm}
\caption{One step in the variational MH random walk \label{alg:variational-MH}}
\begin{algorithmic}[1]
\Require{Current state \(x_n\) of the MH random walk}
\State \(y\sim f(\cdot|x_n)\mathrm d\mu\)
\State \(a\sim \mathcal U([0, 1])\)
\State \(k\gets 1\)
\While {\(a>\rho_k^-(x_n, y)\) and \(a\le \rho_k^+(x_n, y)\)}
  \State \(k \gets k+1\)
\EndWhile
\If{\(a\le \rho_k^-(x_n, y)\)}
  \State \(x_{n+1}\gets y\)
\Else 
  \State \(x_{n+1}\gets x_n\)
\EndIf
\State\Return{\(x_{n+1}\)}
\end{algorithmic}
\end{algorithm}

\subsubsection*{Variational slice sampling}

In the slice sampling we use the unnormalised density twice. The first time when sampling \(y\sim\mathcal U([0, f(x_n)])\) and the second time when checking \(x\in S(y)\) or equivalently \(f(x)\ge y\). For the first problem we note that we surely have \([0, f(x_n)]\subseteq [0, f_1^+(x_n)]\) and hence we can use Algorithm \ref{alg:slice-sample} to sample uniformly from \([0, f(x_n)]\). However, in this algorithm we need to check \(y\in [0,f(x_n)]\) or equivalently \(f(x_n)\ge y\) which is just what we had to do determine whether \(x\in S(y)\). Therefore, it suffices to see how one can check \(f(x)\ge y\) which we will do analogously to the variational MH random walk by gradually tightening the bounds. This yields Algorithm \ref{alg:decide} that returns �TRUE� if \(f(x)\ge y\) and �FALSE� otherwise.

\begin{algorithm}
\caption{Deciding \(f(x)\ge y\) through the bounds \label{alg:decide}}
\begin{algorithmic}[1]
\Require{\(y\in\mathbb R\) and \(x\in\mathcal X\)}
\State \(k\gets 1\)
\While {\(y>f_k^-(x)\) and \(y\le f_k^+(x)\)}
  \State \(k \gets k+1\)
\EndWhile
\If{\(y\le f_k^-(x_n, y)\)}
  \State \Return{TRUE}
\Else 
  \State \Return{FALSE}
\EndIf
\end{algorithmic}
\end{algorithm}

In conclusion we can express both MCMC methods exactly through those bounds as long as the bounds converge. This enables a fast simulation of the Markov chains if the unnormalised density is slow but the bounds \(f_n^\pm\) are easy to compute. In the case that \(f\) is the posterior \eqref{post3} of a DPP such bounds are given in \cite{affandi2014learning} and \cite{bardenet2015inference}.
