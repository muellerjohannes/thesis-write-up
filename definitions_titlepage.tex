\newtheoremstyle{defstyle}
{}              %Space above    
{}              %Space below
{}                      %Body font: original {\normalfont}    
{}                      %Indent amount (empty = no indent,%\parindent = paraindent)    
{\normalfont\bfseries}  %Thm head font original 
{.}{10pt}      
{{\scshape\bfseries \thmnumber{#2}\thmname{ #1}\thmnote{ (#3)}}}
\newtheoremstyle{satzstyle}
{}              %Space above    
{}              %Space below
{\itshape}                      %Body font: original {\normalfont}    
{}                      %Indent amount (empty = no indent,%\parindent = paraindent)    
{\normalfont\bfseries}  %Thm head font original 
{.}{10pt}      
{{\scshape\bfseries \thmnumber{#2} \thmname{ #1}\thmnote{ (#3)}}}
\newtheoremstyle{emptystyle}
{}              %Space above    
{}              %Space below
{}                      %Body font: original {\normalfont}    
{}                      %Indent amount (empty = no indent,%\parindent = paraindent)    
{\scshape\bfseries}  %Thm head font original 
{.}{6pt}      
{{\scshape\bfseries \thmnumber{#2}\thmnote{ #3}}}

\theoremstyle{satzstyle}
\newtheorem{satz}{Satz}[chapter]
\newtheorem{theo}[satz]{Theorem}
\newtheorem{lem}[satz]{Lemma}
\newtheorem{kor}[satz]{Korollar}
\newtheorem{cor}[satz]{Corollary}
\newtheorem{folg}[satz]{Folgerung}
\newtheorem{prop}[satz]{Proposition}
\theoremstyle{defstyle}
\newtheorem{defi}[satz]{Definition}
\newtheorem{bem}[satz]{Bemerkung}
\newtheorem{rem}[satz]{Remark}
\newtheorem{bsp}[satz]{Beispiel}
\newtheorem{ex}[satz]{Example}
\theoremstyle{emptystyle}
\newtheorem{emp}[satz]{}

\renewcommand*{\thesatz}{\arabic{chapter}.\arabic{satz}}
\renewcommand*{\theequation}{\arabic{chapter}.\arabic{equation}}
\renewcommand{\labelenumi}{\textup{(\textit{\roman{enumi}})}}
%\newcommand{\hsp}{\hspace{20pt}}
\titleformat{\chapter}[display]{\huge\bfseries}{Chapter \thechapter}{20pt}{\Huge\bfseries}
\titleformat{\section}{\Large\bfseries}{\thesection}{7pt}{}
\titleformat{\subsection}{\large\bfseries}{\thesubsection}{5pt}{}
\titleformat{\subsubsection}{\large\scshape\bfseries}{\thesubsubsection}{5pt}{}
\renewcommand{\cfttoctitlefont}{\bfseries\Huge}
\renewcommand\cftchapfont{\bfseries}
\renewcommand\thechapter{\textup{\Roman{chapter}}}
%\renewcommand\headmark{\thechapter}

\deftripstyle{custom}{\textsc{\headmark}}{}{\textsc{\pagemark}}{}{}{}

\pagenumbering{roman}

\begin{titlepage}

\begin{center}

\vspace*{2cm}

%\renewcommand{\baselinestretch}{2}

{\Huge Parameter estimation for discrete determinantal point processes}\\[1.5cm]
% Inference for determinantal point processes � from point estimators to a Bayesian setting 

{\Large Dissertation submitted for the degree of}\\[4pt]
{\Large\textsc{Master of Science in Interdisciplinary Mathematics}}\\[24pt]

{\Large Johannes M�ller}\\[12pt]

{\Large \today}\\[20pt]

\vfill
\includegraphics[width=0.25\textwidth]{figures/crest_black}\\[1cm]

\textsc{Supervised by}\\[.1cm]
\textsc{\Large Professor Nikolaos Zygouras and Dr Theodoros Damoulas}\\[.6cm]
%\includegraphics[width=0.45\textwidth]{./the_warwick_uni_black}\\[.3cm]
\textsc{\huge University of Warwick}\\[.1cm]
\textsc{\LARGE Department of Mathematics}\\[3.5cm]

\end{center}
\end{titlepage}

\cleardoublepage