%\chapter{Introduction and motivating examples}
\chapter*{Preface}
%\addtocontents{toc}{Preface}
\addcontentsline{toc}{chapter}{Preface}

%\section{Motivation}

%\todo{explain why DPPs are awesome}

%\section{Previous work}

%\section{Aim and outline of the dissertation}

\begin{enumerate}
\item Outline:
\begin{enumerate}
\item Chapter I: Basics that we need to investigate the task of parameter estimation including the existence and simulation of DPPs and some toy examples.
\item Chapter II: Presentation of two different point estimators including the proof of consistency of both.
\item Chapter III: Discussion of a Bayesian framework for parameter estimation in general and for DPPs specifically including the benefits of this approach.
\end{enumerate}
\item Contributions:
\begin{enumerate}
\item Give a self contained presentation of the estimator for the marginal kernel that is presented in ... including a different, arguably easier proof of the consistency
\item Show that the MLEs for different parameters exist with increasing probability and are in fact consistent.
\item Give a short introduction into Bayesian parameter estimation and MCMC methods with a presentation of the mathematical foundations of those.
\item Provide easy examples throughout the thesis including simulation and parameter estimations for those. The code for this is also included in the appendix.
\end{enumerate}
\item It is the aim to give a mostly self contained approach to the topic that is accessible to any student familiar with basic notions of linear algebra, analysis and probability theory. We proof almost everything we use in this dissertation or give precise references if the statements are not (mathematical) general knowledge.
\end{enumerate}