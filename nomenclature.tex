%\manualmark \markboth{}{Nomenclature}
\nomenclature{\(\operatorname{sgn}\)}{Either the sign of a number or the parity of a permutation.}

\nomenclature{\(\mathbb Z_2\)}{The cyclic group consisting of \(\left\{ 0, 1\right\}\) with the addition modulo \(2\).}

\nomenclature{\(\mathbb F_2\)}{The finite field \(\left\{ 0, 1 \right\}\) with the addition and multiplication modulo \(2\).}

\nomenclature{\(\mathbb N\)}{The natural numbers.}

\nomenclature{\(\mathbb R\)}{The real numbers.}

\nomenclature{\(D^2f\)}{Hessian matrix, i.e. second derivative of a function \(f\colon\mathbb R^d\to \mathbb R\).}

\nomenclature{\(\mathbb R^d\)}{The Euclidean \(d\)-dimensional space with Euclidean norm \(\left\lVert \cdot \right\rVert\).}

\nomenclature{\(L^2(\mu)\)}{The space of square integrable functions with scalar product \[(\phi, \psi)_{L^2(\mu)} = \int\phi(x)\psi(x)\mu(\mathrm dx).\]}

\nomenclature{\(\mathbb R^{N\times N}_{\text{sym}, +}\)}{The set of non negative definite symmetric \(N\times N\) matrices.}

\nomenclature{\(\mathbb R_+\)}{The set of non negative real numbers \([0, \infty)\).}

\nomenclature{\(O(g(x))\)}{We write \(f(x) = O(g(x))\) if \(f(x) \le M g(x)\) for all \(x\ge x_0\) and one \(M>0\).}

\nomenclature{\(\operatorname{span}\)}{This denotes the span of a collection of vectors.}

\nomenclature{\(A\ge B\)}{We write \(A\ge0\) if \(A\) is non negative definite and \(A\ge B\) if \(A-B\ge 0\) where \(A\) and \(B\) are symmetric matrices.}

\nomenclature{\(\arg\max\)}{The \(\arg\max\) function selects one arbitrary maximiser given it exists.}

\nomenclature{\(S_n\)}{The permutation group of \(\left\{ 1, \dots, n\right\}\) or other sets with \(n\) elements.}

\nomenclature{\(x\gets y\)}{This denotes the assignment of \(y\) to the variable \(x\) in pseudocode.}

\nomenclature{\(\mathcal U(S)\)}{The uniform distribution on a set \(S\) with respect to some measure that should be clear from the context.}

\nomenclature{\(\mathcal E(\alpha)\)}{The exponential distribution with parameter \(\alpha>0\) given by the density \(\mathds{1}_{[0, \infty)}(s)\alpha e^{-\alpha s}\).}

\nomenclature{\(x\sim \mathbb P\)}{This denotes that \(x\) is a realisation of a random variable \(X\) with law \(\mathbb P\).}