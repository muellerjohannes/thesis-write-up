\begin{center}
{\LARGE\textsc{Acknowledgements}\\[.9cm]}
% \normalsize
\begin{minipage}{11cm}
I would like to thank the dreamteam for being such an eggcelent friendgroup and for all the pun we had during the last year.
\end{minipage}
\end{center}

\clearpage


\begin{center}
{\LARGE\textsc{Abstract}\\[.9cm]}
\begin{minipage}{11cm}
Determinantal point processes are random subsets that exhibit a diversifying behaviour in the sense that the randomly selected points tend to be not similar in some way. This repellent structure first arrose in theortical physics and pure mathematics, but they have recently been used to model a variety of many real world scenarios in a machine learning setup. We aim to give an overview over the main ideas of this approach which is easily accessible even without prior knowledge in the area of machine learning and sometimes omit technical calculations in order to keep the focus on the concepts.
\end{minipage}
\end{center}

\clearpage
